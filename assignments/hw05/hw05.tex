\documentclass[]{book}

%These tell TeX which packages to use.
\usepackage{array,epsfig}
\usepackage{amsmath}
\usepackage{amsfonts}
\usepackage{amssymb}
\usepackage{amsxtra}
\usepackage{amsthm}
\usepackage{mathrsfs}
\usepackage{color}
\usepackage[spanish, mexico]{babel}
\usepackage[utf8]{inputenc}
\usepackage{hyperref}

%Here I define some theorem styles and shortcut commands for symbols I use often
\theoremstyle{definition}
\newtheorem{defn}{Definition}
\newtheorem{thm}{Theorem}
\newtheorem{cor}{Corollary}
\newtheorem*{rmk}{Remark}
\newtheorem{lem}{Lemma}
\newtheorem*{joke}{Joke}
\newtheorem{ex}{Example}
\newtheorem*{sol}{Solution}
\newtheorem{prop}{Proposition}

\newcommand{\lra}{\longrightarrow}
\newcommand{\ra}{\rightarrow}
\newcommand{\surj}{\twoheadrightarrow}
\newcommand{\graph}{\mathrm{graph}}
\newcommand{\bb}[1]{\mathbb{#1}}
\newcommand{\Z}{\bb{Z}}
\newcommand{\Q}{\bb{Q}}
\newcommand{\R}{\bb{R}}
\newcommand{\C}{\bb{C}}
\newcommand{\N}{\bb{N}}
\newcommand{\M}{\mathbf{M}}
\newcommand{\m}{\mathbf{m}}
\newcommand{\MM}{\mathscr{M}}
\newcommand{\HH}{\mathscr{H}}
\newcommand{\Om}{\Omega}
\newcommand{\Ho}{\in\HH(\Om)}
\newcommand{\bd}{\partial}
\newcommand{\del}{\partial}
\newcommand{\bardel}{\overline\partial}
\newcommand{\textdf}[1]{\textbf{\textsf{#1}}\index{#1}}
\newcommand{\img}{\mathrm{img}}
\newcommand{\ip}[2]{\left\langle{#1},{#2}\right\rangle}
\newcommand{\inter}[1]{\mathrm{int}{#1}}
\newcommand{\exter}[1]{\mathrm{ext}{#1}}
\newcommand{\cl}[1]{\mathrm{cl}{#1}}
\newcommand{\ds}{\displaystyle}
\newcommand{\vol}{\mathrm{vol}}
\newcommand{\cnt}{\mathrm{ct}}
\newcommand{\osc}{\mathrm{osc}}
\newcommand{\LL}{\mathbf{L}}
\newcommand{\UU}{\mathbf{U}}
\newcommand{\support}{\mathrm{support}}
\newcommand{\AND}{\;\wedge\;}
\newcommand{\OR}{\;\vee\;}
\newcommand{\Oset}{\varnothing}
\newcommand{\st}{\ni}
\newcommand{\wh}{\widehat}

%Pagination stuff.
\setlength{\topmargin}{-.3 in}
\setlength{\oddsidemargin}{0in}
\setlength{\evensidemargin}{0in}
\setlength{\textheight}{9.in}
\setlength{\textwidth}{6.5in}
\setlength{\itemsep}{0.45in}
\pagestyle{empty}



\begin{document}

\begin{center}
{\huge Implementación de Métodos Computacionales TC2037}\\[1.5ex]
{\large \textbf{Tarea 05}}\\[1.5ex] %You should put your name here
\end{center}

\vspace{0.2 cm}

\subsection*{Gramáticas Libres de Contexto y Autómatas de Pila}

\textbf{Esta tarea es en equipos} y es mero trámite porque está bien \textit{EZ} mode.
\vspace{1cm}

Con ayuda de JFLAP\footnote{Para el JFLAP (\url{http://www.jflap.org/}): Seleccionar \textit{Pushdown Automata}, \textit{Multiple character input}. Las acciones deben introducirse de la forma $a,S;P$, donde $a$ es lo que se lee en la \textdf{cinta}, $S$ lo que se \textbf{quita} de la pila y $P$ lo que se \textbf{pone} en la pila.} o bien utilizando Automaton Simulator\footnote{Para el Automaton Simulator (\url{https://automatonsimulator.com/}): Seleccionar \textit{PDA}. Las acciones deben introducirse en los inputs disponibles con la forma $a, S, P$, donde $a$ es lo que se lee en la \textdf{cinta}, $S$ lo que se \textbf{quita} de la pila y $P$ lo que se \textbf{pone} en la pila.}, generen \textdf{un AP correcto y completo} para cada uno de los lenguajes especificados.
Además, genera una Gramática Libre de Contexto para cada uno \textdf{en un documento} en \textit{typesetting} (PDF via \LaTeX o Word). Las presentaciones tienen información sobre esto que les pueden ser de utilidad.

\begin{enumerate}
    \item El lenguaje en $\{a,b\}^*$ de palíndromos (palabras que se leen igual de derecha a izquierda y viceversa).
    \item El lenguaje en $\{0,1\}^*$ de palabras de la forma $\{0^n 1^n : n \in \mathbb{N}\cup \{0\}\}$
\end{enumerate}

\bigskip

Suban los dos APs (JSON o JFF) y las GLCs en un documento en \textit{typesetting} (PDF) como entregable.
¡Con que un integrante del equipo lo suba es más que suficiente!
\end{document}