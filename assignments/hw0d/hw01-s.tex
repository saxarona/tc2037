\documentclass[]{book}

%These tell TeX which packages to use.
\usepackage{array,epsfig}
\usepackage{amsmath}
\usepackage{amsfonts}
\usepackage{amssymb}
\usepackage{amsxtra}
\usepackage{amsthm}
\usepackage{mathrsfs}
\usepackage{color}
\usepackage[spanish, mexico]{babel}
\usepackage[utf8]{inputenc}

%Here I define some theorem styles and shortcut commands for symbols I use often
\theoremstyle{definition}
\newtheorem{defn}{Definition}
\newtheorem{thm}{Theorem}
\newtheorem{cor}{Corollary}
\newtheorem*{rmk}{Remark}
\newtheorem{lem}{Lemma}
\newtheorem*{joke}{Joke}
\newtheorem{ex}{Example}
\newtheorem*{sol}{Solution}
\newtheorem{prop}{Proposition}

\newcommand{\lra}{\longrightarrow}
\newcommand{\ra}{\rightarrow}
\newcommand{\surj}{\twoheadrightarrow}
\newcommand{\graph}{\mathrm{graph}}
\newcommand{\bb}[1]{\mathbb{#1}}
\newcommand{\Z}{\bb{Z}}
\newcommand{\Q}{\bb{Q}}
\newcommand{\R}{\bb{R}}
\newcommand{\C}{\bb{C}}
\newcommand{\N}{\bb{N}}
\newcommand{\M}{\mathbf{M}}
\newcommand{\m}{\mathbf{m}}
\newcommand{\MM}{\mathscr{M}}
\newcommand{\HH}{\mathscr{H}}
\newcommand{\Om}{\Omega}
\newcommand{\Ho}{\in\HH(\Om)}
\newcommand{\bd}{\partial}
\newcommand{\del}{\partial}
\newcommand{\bardel}{\overline\partial}
\newcommand{\textdf}[1]{\textbf{\textsf{#1}}\index{#1}}
\newcommand{\img}{\mathrm{img}}
\newcommand{\ip}[2]{\left\langle{#1},{#2}\right\rangle}
\newcommand{\inter}[1]{\mathrm{int}{#1}}
\newcommand{\exter}[1]{\mathrm{ext}{#1}}
\newcommand{\cl}[1]{\mathrm{cl}{#1}}
\newcommand{\ds}{\displaystyle}
\newcommand{\vol}{\mathrm{vol}}
\newcommand{\cnt}{\mathrm{ct}}
\newcommand{\osc}{\mathrm{osc}}
\newcommand{\LL}{\mathbf{L}}
\newcommand{\UU}{\mathbf{U}}
\newcommand{\support}{\mathrm{support}}
\newcommand{\AND}{\;\wedge\;}
\newcommand{\OR}{\;\vee\;}
\newcommand{\Oset}{\varnothing}
\newcommand{\st}{\ni}
\newcommand{\wh}{\widehat}

%Pagination stuff.
\setlength{\topmargin}{-.3 in}
\setlength{\oddsidemargin}{0in}
\setlength{\evensidemargin}{0in}
\setlength{\textheight}{9.in}
\setlength{\textwidth}{6.5in}
\setlength{\itemsep}{0.45in}
\pagestyle{empty}



\begin{document}

\begin{center}
{\huge Matemáticas Computacionales TC2020 -- N}\\[1.5ex]
{\large \textbf{Tarea 0d -- Soluciones}\\[1.5ex] %You should put your name here
} %You should write the date here.
\end{center}

\vspace{0.2 cm}

\subsection*{Preliminares: Conjuntos, relaciones y funciones}

\begin{enumerate}
	\itemsep0.35in
	\item Calcula el resultado de las operaciones siguientes:
	\begin{enumerate}
        \item $\{a,b,c\} \times \{1,2,3,4\} = \{(a,1), (a,2), (a,3), (a,4), (b,1), (b,2), (b,3), (b,4), (c,1), (c,2), (c,3), (c,4)\}$
		\item $\{a,\{b\}, \{\{c\}\}\} \times \varnothing = \varnothing$
		\item $\mathscr{P}(\{x : x \in \N, x < 4\}) = \mathscr{P}(\{1,2,3\}) = \{\varnothing, \{1\}, \{2\}, \{3\}, \{1,2\}, \{1,3\}, \{2,3\}, \{1,2,3\}\}$
		\item $|\mathscr{P}(\{y : y \in \Z, 0 < y < 10 \})| = |\mathscr{P}(\{1,2,3,4,5,6,7,8,9\})| = 2^9 = 512$
	\end{enumerate}

	\item Las siguientes son relaciones de $\{1,2,3,4\}$ a $\{1,2,3,4\}$.
	Indica cuáles de ellas son relaciones \textdf{transitivas}, \textdf{reflexivas} o \textdf{simétricas}.
	Indica también cuáles son \textdf{funciones} y cuáles son sólo \textdf{relaciones}.
	En caso de ser funciones, indica si son funciones \textdf{totales} o \textdf{parciales}, y cuáles son \textdf{inyectivas}, \textdf{sobreyectivas} y cuáles son \textdf{biyectivas}.
	\begin{enumerate}
        \item $\{(2,2), (3,3), (1,1), (4,4)\}$\\
        Reflexiva, simétrica, transitiva, función total, inyectiva, sobreyectiva y por tanto biyectiva.
        \item $\{(1,1), (2,2), (3,3), (4,3)\}$\\
        No reflexiva, no simétrica, transitiva, función total, no inyectiva, no sobreyectiva, no biyectiva.
        \item $\{(1,1), (3,4), (2,2), (3,3)\}$\\
        No reflexiva, no simétrica, transitiva, no es función, sólo es relación (así que tampoco es inyectiva ni sobreyectiva ni biyectiva).
        \item $\{(1,1), (2,2), (3,3)\}$\\
        No reflexiva, simétrica, transitiva, función parcial, inyectiva, no sobreyectiva (y por tanto no biyectiva).
	\end{enumerate}

	\item Apóyate en la información vista en clase e investiga qué es la \textdf{cerradura transitiva}. Posteriormente escribe su definición con tus propias palabras. Luego, encuentra la cerradura transitiva de cada una de las relaciones del problema anterior.\footnotemark
\end{enumerate}

    La cerradura transitiva $R'$ es la relación transitiva más pequeña que incluye a la relación original $R$: $R \subseteq R'$. Como quien dice, le agregamos a $R$ los elementos necesarios para hacerla transitiva, y entonces obtenemos la cerradura transitiva $R'$.

    Como todas las relaciones del ejercicio anterior son transitivas, entonces la relación transitiva más pequeña que contiene a cada una de ellas es esa misma relación.

\footnotetext{No olvides citar tus fuentes de manera adecuada. Considera que sean fuentes fiables y, de ser posible, lista dos o tres recursos; no te quedes con la primera definición que encuentres.}
\end{document}


