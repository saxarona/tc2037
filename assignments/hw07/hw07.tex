\documentclass[]{book}

%These tell TeX which packages to use.
\usepackage{array,epsfig}
\usepackage{amsmath}
\usepackage{amsfonts}
\usepackage{amssymb}
\usepackage{amsxtra}
\usepackage{amsthm}
\usepackage{mathrsfs}
\usepackage{color}
\usepackage[spanish, mexico]{babel}
\usepackage[utf8]{inputenc}
\usepackage{hyperref}
\usepackage{tcolorbox}
\usepackage{minted}

%Here I define some theorem styles and shortcut commands for symbols I use often
\theoremstyle{definition}
\newtheorem{defn}{Definition}
\newtheorem{thm}{Theorem}
\newtheorem{cor}{Corollary}
\newtheorem*{rmk}{Remark}
\newtheorem{lem}{Lemma}
\newtheorem*{joke}{Joke}
\newtheorem{ex}{Example}
\newtheorem*{sol}{Solution}
\newtheorem{prop}{Proposition}

\newcommand{\lra}{\longrightarrow}
\newcommand{\ra}{\rightarrow}
\newcommand{\surj}{\twoheadrightarrow}
\newcommand{\graph}{\mathrm{graph}}
\newcommand{\bb}[1]{\mathbb{#1}}
\newcommand{\Z}{\bb{Z}}
\newcommand{\Q}{\bb{Q}}
\newcommand{\R}{\bb{R}}
\newcommand{\C}{\bb{C}}
\newcommand{\N}{\bb{N}}
\newcommand{\M}{\mathbf{M}}
\newcommand{\m}{\mathbf{m}}
\newcommand{\MM}{\mathscr{M}}
\newcommand{\HH}{\mathscr{H}}
\newcommand{\Om}{\Omega}
\newcommand{\Ho}{\in\HH(\Om)}
\newcommand{\bd}{\partial}
\newcommand{\del}{\partial}
\newcommand{\bardel}{\overline\partial}
\newcommand{\textdf}[1]{\textbf{\textsf{#1}}\index{#1}}
\newcommand{\img}{\mathrm{img}}
\newcommand{\ip}[2]{\left\langle{#1},{#2}\right\rangle}
\newcommand{\inter}[1]{\mathrm{int}{#1}}
\newcommand{\exter}[1]{\mathrm{ext}{#1}}
\newcommand{\cl}[1]{\mathrm{cl}{#1}}
\newcommand{\ds}{\displaystyle}
\newcommand{\vol}{\mathrm{vol}}
\newcommand{\cnt}{\mathrm{ct}}
\newcommand{\osc}{\mathrm{osc}}
\newcommand{\LL}{\mathbf{L}}
\newcommand{\UU}{\mathbf{U}}
\newcommand{\support}{\mathrm{support}}
\newcommand{\AND}{\;\wedge\;}
\newcommand{\OR}{\;\vee\;}
\newcommand{\Oset}{\varnothing}
\newcommand{\st}{\ni}
\newcommand{\wh}{\widehat}

%Pagination stuff.
\setlength{\topmargin}{-.3 in}
\setlength{\oddsidemargin}{0in}
\setlength{\evensidemargin}{0in}
\setlength{\textheight}{9.in}
\setlength{\textwidth}{6.5in}
\setlength{\itemsep}{0.45in}
\pagestyle{empty}



\begin{document}

\begin{center}
{\huge Implementación de Métodos Computacionales TC2037}\\[1.5ex]
{\large \textbf{Tarea 07}\\[1.5ex] %You should put your name here
} %You should write the date here.
\end{center}

\vspace{0.2 cm}

\section*{Programación Lógica}

\textbf{Esta tarea es en equipos (Season II)}.
\vspace{4ex}

Implementen un script en Prolog para resolver los siguientes problemas. No es necesario que hagan 

\subsection*{Cena de gala}

El consejo de la ciudad hará una cena de gala. Sin embargo, algunos de los invitados tienen necesidades especiales y por lo mismo la recepción y la asignación de asientos debe ser sumamente cuidadosa.

Con tres mesas disponibles (una de 5 asientos, y dos más con 6 asientos), el consejo de la ciudad busca evitar conflictos de intereses que están dados por las siguientes restricciones:

\begin{itemize}
    \item La familia del Faraón (papá, mamá y dos hijos) tiene que estar junta, y no puede estar cerca del Sacerdote. Los niños están intrigados por el Robot, así que quieren sentarse en la misma mesa para inspeccionarlo mejor.
    \item Demis, Vangelis y Mikis (``los Griegos'') necesitan estar en la misma mesa.
    \item Wilson y Akerfeldt quieren sentarse con los griegos, pero no cerca de Parsons.
    \item Parsons y Gilmour preferirían sentarse juntos.
    \item Dickinson, Harris, Dio y Summers---los \textit{metalheads}---son buena onda, así que no tienen problema alguno. Sólo quieren cerveza.
    \item El Sacerdote prefiere estar lejos de los \textit{metalheads} y del Robot.
\end{itemize}

Para este problema se sugiere que utilicen variables, y que consideren una restricción extra que considere la suma de los valores de cada variable como un número exacto. De este modo pueden representar las tres mesas con sus respectivos espacios.

\pagebreak

\subsection*{K-pop Recommender System}

Los sistemas de recomendación de contenidos digitales se basan mayormente en modelos probabilísticos que entrenan con lo que los usuarios buscan y recomiendan como contenido similar.
Sin embargo, existe otro enfoque basado en razonamiento en el cual se generan reglas que analizan el contenido mismo, para decidir qué recomendaciones hacer si dos artistas contienen características similares.

Implementaremos la lógica detrás de un sistema basado en razonamiento para recomendación de grupos de K-Pop, considerando la siguiente información:

\vspace{2ex}

A los fans de Girls' Generation se les conoce como \textit{SONEs}, mientras que a los de Red Velvet les llaman \textit{Reveluvs} y a los de Blackpink, \textit{Blinks}.
A todos los \textit{SONEs} les gusta Red Velvet o Blackpink.
Si les gusta Red Velvet, entonces le gustan las baladas.
Si les gusta Blackpink, entonces les gusta el dance.
Si les gusta la música electrónica y son \textit{SONEs}, recomiéndales DJ Hyo.
La gente que disfruta el dance y las baladas, consideran a Chung Ha como una buena recomendación.
Todos los \textit{SONEs} a los que les gusta el drama, calificaron a Seohyun como una buena recomendación.
Taeyeon le interesa a todos los usuarios que gustan de las baladas y son \textit{SONEs}.
Si les gusta el drama y las baladas, es seguro que Heize les gustará.

\bigskip

\begin{itemize}
    \item Generen la base de conocimiento proponiendo el vocabulario que usará su sistema y convirtiendo las reglas a forma simbólica
    \item Usando prolog, averigüen si a un usuario nuevo que es fan de las Girls' Generation y es \textit{Reveluv}, le gustará Taeyeon
    \item Considerando al mismo usuario, demuestren o refuten que le gustará Heize
    \item ¿Si un usuario es SONE, \textit{Reveluv}, \textit{Blink} y le gusta el drama, cuáles son las posibles recomendaciones que le hará el sistema?
\end{itemize}

Para este problema se sugiere que utilicen una función llamada \texttt{discontiguous} al principio de su código fuente.
Esta función indica al script que no todos los predicados están dados de alta. Se usa de la siguiente manera:

\begin{tcolorbox}
    \mintinline{prolog}{:- discontiguous(likes/2).}
\end{tcolorbox}

En este ejemplo, \texttt{likes} es el nombre de su cláusula, y \texttt{/2} es el número de parámetros que recibe la función.

\vspace{7ex}

Suban su código fuente (.pl) en Canvas. Con que un integrante del equipo suba el archivo es más que suficiente.
\end{document}