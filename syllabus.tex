\documentclass[12pt, letterpaper, oneside]{article}
%\usepackage{geometry}
\usepackage[spanish, mexico]{babel}
\usepackage[utf8]{inputenc}
\usepackage{amssymb}
\usepackage[inner=1.5cm,outer=1.5cm,top=2.5cm,bottom=2.5cm]{geometry}
\pagestyle{empty}
\usepackage{graphicx}
\usepackage{fancyhdr, lastpage, bbding, pmboxdraw}
\usepackage[usenames,dvipsnames]{color}
\definecolor{darkblue}{rgb}{0,0,.6}
\definecolor{darkred}{rgb}{.7,0,0}
\definecolor{darkgreen}{rgb}{0,.6,0}
\definecolor{red}{rgb}{.98,0,0}
\usepackage[colorlinks,pagebackref,pdfusetitle,urlcolor=darkblue,citecolor=darkblue,linkcolor=darkred,bookmarksnumbered,plainpages=false]{hyperref}
\renewcommand{\thefootnote}{\fnsymbol{footnote}}

\newcommand{\thecourse}{Implementación de Métodos Computacionales (TC2037--15)}
\newcommand{\thesemester}{Febrero--Junio 2021}
\newcommand{\theinstructor}{Xavier F. C. Sánchez Díaz}
\newcommand{\themail}{sax@tec.mx}
\newcommand{\thetime}{Lu Ju 15:00--17:00 hr}
\newcommand{\theplace}{Zoom Meetings}

\newcommand{\topic}{{\color{darkgreen}{\Rectangle}}}
\newcommand{\subtopic}{{\enskip \color{darkblue}{\Rectangle}}}

\setlength{\headheight}{14.5pt}

\pagestyle{fancyplain}
\fancyhf{}
\lhead{ \fancyplain{}{\thecourse} }
%\chead{ \fancyplain{}{} }
\rhead{ \fancyplain{}{\thesemester} }
%\rfoot{\fancyplain{}{page \thepage\ of \pageref{LastPage}}}
\fancyfoot[RO] {Página \thepage\ de \pageref{LastPage}}
\thispagestyle{plain}

%%%%%%%%%%%% LISTING %%%
\usepackage{listings}
\usepackage{caption}
\DeclareCaptionFont{white}{\color{white}}
\DeclareCaptionFormat{listing}{\colorbox{gray}{\parbox{\textwidth}{#1#2#3}}}
% \captionsetup[lstlisting]{format=listing,labelfont=white,textfont=white}
\usepackage{verbatim} % used to display code
\usepackage{fancyvrb}
\usepackage{acronym}
\usepackage{amsthm}
\VerbatimFootnotes % Required, otherwise verbatim does not work in footnotes!



\definecolor{OliveGreen}{cmyk}{0.64,0,0.95,0.40}
\definecolor{CadetBlue}{cmyk}{0.62,0.57,0.23,0}
\definecolor{lightlightgray}{gray}{0.93}
%%%%%%%%%%%%%%%%%%%%%%%%%%%%%%%%%%%%

\begin{document}
  \begin{center}
  {\Large \textsc{\thecourse}}
  \end{center}
  \begin{center}
  \thesemester
  \end{center}

  \begin{center}
  \rule{6in}{0.4pt}
  \begin{minipage}[t]{.8\textwidth}
  \begin{tabular}{llcccll}
  \textbf{Instructor:} & \theinstructor & & &  & \textbf{Hora:} & \thetime \\
  \textbf{Email:} &  \href{mailto:sax@tec.mx}{\themail} & & & & \textbf{Lugar:} & \theplace
  \end{tabular}
  \end{minipage}
  \rule{6in}{0.4pt}
  \end{center}
  \vspace{.5cm}
  \setlength{\unitlength}{1in}
  \renewcommand{\arraystretch}{2}

  \noindent\textbf{Página del curso:}
  
  \begin{enumerate}
  \item \url{https://saxarona.github.io/teaching/tc2037}
  \end{enumerate}

  \vskip.15in

  \noindent\textbf{Horario de oficina:}
  Revisa el calendario y agenda una cita \href{https://calendar.google.com/calendar/u/0/selfsched?sstoken=UUFXb0ZNNmJKZFZLfGRlZmF1bHR8OTk5YjNjMzk4ZGE5NTQ2YWU2YmEwZTNjMGE3NTc1Y2I}{aquí}. Los horarios de asesoría dependen de la carga de trabajo y varían entre períodos.
  Lo mejor es mandar un correo y ponernos de acuerdo.
  Todos los horarios son en TCM (Tiempo del Centro de México).

  \vskip.15in

  \noindent\textbf{Material recomendado:} % \footnotemark
  Ésta es una lista de recursos que pueden serte de utilidad durante el curso.

  \begin{itemize}
    \item \textit{Introduction to Theory of Computation} es un libro de texto de libre distribución de Anil Maheshwari y Michiel Smid, de Carleton University. Disponible \href{https://cglab.ca/~michiel/TheoryOfComputation/TheoryOfComputation.pdf}{aquí}.
    \item S. Krishnamurthi, \textit{Programming Languages: Application and Interpretation}, Providence, RI: Brown University, 2017. Disponible \href{http://cs.brown.edu/~sk/Publications/Books/ProgLangs/2007-04-26/}{aquí}.
    \item M. Lipova\u{c}a, \textit{Learn you a Haskell}, 2020. Disponible \href{http://learnyouahaskell.com/}{aquí}.
    \item Haskell Wikibook. Disponible \href{https://en.wikibooks.org/wiki/Haskell}{aquí}.
  \end{itemize}

  \vskip.15in

  \noindent\textbf{Objetivos:}
  Al final del curso, el alumno:

  \begin{itemize}
    \item Implementará algoritmos computacionales que solucionan problemas.
    \item Podrá optimizar algoritmos computacionales que se aplican en el desarrollo de soluciones.
    \item Generará modelos computacionales para la solución de problemas.
    \item Implementará modelos computacionales en la solución numérica de un problema.
  \end{itemize}

  \vskip.15in
  \noindent\textbf{Requisitos:}
  Haber cursado Modelación de la Ingeniería con Matemática Computacional (TC1003B) y Programación de Estructuras de Datos y Algoritmos Fundamentales (TC1031).

  \vspace*{.15in}
  \noindent \textbf{Índice analítico del curso:}
  El curso es una mezcla de dos grandes asignaturas clásicas: teoría de la computación y lenguajes de programación. Tras hacer una fusión, los módulos son los siguientes:

  \begin{center} 
  \begin{minipage}{5in}
  \begin{flushleft}
  {\large I. Teoría de la computación} \\[2ex]
  \topic ~Teoría de la computación \\
  \subtopic ~Conceptos matemáticos preliminares \\
  \subtopic ~Conceptos de lenguajes de programación y jerarquía de Chomsky \\
  \subtopic ~Diseño formal de lenguajes de programación \\
  \subtopic ~Paradigmas de lenguajes de programación \\
  \topic ~Lenguajes regulares \\
  \subtopic ~Autómatas finitos deterministas \\
  \subtopic ~Autómatas finitos no deterministas \\
  \subtopic ~Expresiones regulares \\
  \subtopic ~Equivalencia entre autómatas \\
  \subtopic ~Gramáticas regulares \\
  \subtopic ~Análisis léxico \\
  \topic ~Lenguajes libres de contexto \\
  \subtopic ~Gramáticas libres de contexto \\
  \subtopic ~Autómatas de pila (AP) \\
  \topic ~Equivalencias entra APs y GLCs \\
  \subtopic ~Análisis sintáctico \\
  \topic ~Máquinas de Turing (MT)\\
  \subtopic ~Definición de MTs \\
  \subtopic ~MTs como generadoras de lenguajes \\
  \subtopic ~Teoría de decibilidad \\
  \subtopic ~Teoría de computabilidad \\[2.5ex]
  {\large II. Lenguajes de Programación} \\[2ex]
  \topic ~Programación funcional \\
  \subtopic ~Introducción al cálculo lambda \\
  \subtopic ~Lenguajes Funcionales \\
  \subtopic ~Recursión como base del control de flujo \\
  \subtopic ~Listas como esencia en el manejo de datos \\
  \topic ~Programación Lógica \\
  \subtopic ~Características generales de la programación lógica \\
  \subtopic ~Principios de la programación lógica \\
  \subtopic ~Lenguajes de programación lógica \\
  \subtopic ~Modelo de ejecución basado en unificación y backtracking \\
  \topic ~Programación concurrente y paralela \\
  \subtopic ~Características generales de la programación concurrente y paralela \\
  \subtopic ~Administración de threads, procesos y algoritmos de planificación \\
  \subtopic ~Eventos y señales y algoritmos de sincronización \\
  \subtopic ~Lenguajes concurrentes \\
  \subtopic ~Modelos de manejo de concurrencia \\
  \subtopic ~Programación multinúcleo \\

  \end{flushleft}
  \end{minipage}
  \end{center}

  \vspace*{.15in}
  \noindent\textbf{Política de evaluación:}
  Tareas formativas (47\%), Tareas con evidencia de competencia (53\%).
  \vspace*{.15in}
  % La suma será posteriormente multiplicada por 95\% debido a que el 5\% restante corresponde a la Semana i.

  \noindent\textbf{Recuerda que lo que se evalúa es tu desempeño, no tu persona}.
  En los exámenes, evaluamos lo que escribes, no lo que piensas ni lo que sabes.
  Las evaluaciones---a pesar de sus limitaciones---son un elemento básico para que la institución pueda certificar, al final de tu carrera, que asististe a los cursos y que posees los conocimientos, habilidades, actitudes y valores de un profesionista.

  % \vskip.15in
  % \noindent\textbf{Fechas importantes:}
  % Esta materia no tiene exámenes.
  % Sin embargo, es importante que recuerdes cuándo son nuestr:

  % \begin{center} \begin{minipage}{3.8in}
  % \begin{flushleft}
  % Examen 1 \dotfill ~21 de septiembre \\
  % % Semana \textit{i} \dotfill ~18 al 24 de septiembre\\
  % Examen 2 \dotfill ~02 de noviembre \\
  % Examen final \dotfill ~30 de noviembre
  % \end{flushleft}
  % \end{minipage}
  % \end{center}
  % \pagebreak
  \vskip.15in
  \noindent\textbf{Políticas del curso:}  
  \begin{itemize}
  \item Se sugiere que al inicio del semestre navegues por la página del curso, el curso en Canvas y que revises los contenidos, su forma de evaluación y las reglas. \textbf{El desconocimiento de una regla que fue dada a conocer no justifica su omisión}.
  \item Verifica que tu correo del Tec esté funcionando, ya que será utilizado como \textbf{medio oficial de comunicación}. El hecho de que no tengas acceso a tu correo no es justificación para no llevar a cabo una entrega.
  \item Las tareas serán entregadas por el medio especificado y antes de la fecha límite. En caso de que no puedas entregar una tarea a tiempo, es probable que puedas entregarla de otro modo aunque con una penalización. Acércate al profesor.
  \item En caso de las tareas que contribuyen a las \textbf{evidencias}, la fecha de entrega no se podrá posponer salvo en condiciones excepcionales y por fuerzas de causa mayor.
  \item Las soluciones a las tareas deberán ser entregadas en limpio y en el formato especificado. Si son de texto, un archivo \textit{typeset} (nativamente digital, hecho en \textit{Word} o \LaTeX), en un archivo PDF y subidas a la plataforma. Si son de programación, archivos fuente (\texttt{.rkt}, \texttt{.hs}, \texttt{.pl}, \texttt{.erl}). Evita subir fotos o \textit{scans} de trabajos a mano o screenshots de tu código.
  \item Para tareas en las que la solución sea de más de un archivo, sube una carpeta comprimida en formato ZIP.
  % \item Las soluciones a las tareas con un puntaje casi perfecto podrían ser consideradas como soluciones oficiales de dicha tarea y subidas a la plataforma. En caso de ser así, el estudiante ganará puntos extras.
  \item Si hay algo que crees necesario que deba tomar en cuenta al momento de calificar tu tarea, escríbelo en los comentarios de la plataforma, o bien crea un archivo de texto con el nombre \texttt{README} y escribe ahí tu mensaje e inclúyelo en el archivo comprimido. No envíes estos mensajes por correo.
  \item Puedes discutir los problemas de la tarea con otros estudiantes, pero recuerda que debes subir un archivo escrito por ti (y los miembros de tu equipo, según sea el caso). En trabajos colaborativos, un solo entregable basta, pero asegúrate de incluir a todos los integrantes.
  \item Las aclaraciones de los alumnos respecto a calificaciones de actividades y exámenes sólo podrán hacerse dentro de las dos semanas siguientes a la publicación de las calificaciones respectivas.
  % \item Los comentarios o aclaraciones que haga el profesor durante la aplicación de un examen son usados por el alumno bajo su propia responsabilidad, si considera que le son de utilidad, y en ningún momento podrán usarse como argumento para discutir la calificación de algún problema del examen.
  % \item En caso de que un alumno no pueda presentar una evidencia por causas de fuerza mayor, deberá conseguir un visto bueno de la dirección de carrera, quien mandará un correo u otro documento equivalente al profesor. El profesor no revisará directamente comprobantes médicos o documentos de esa índole.
  \end{itemize}

  % \vskip.15in
  \pagebreak
  \noindent\textbf{Políticas de las sesiones en línea:}  
  \begin{itemize}
  \item La entrada a la reunión de ZOOM es a la hora especificada.
  \item Las actividades desarrolladas durante una sesión a la que no asististe no se repondrán salvo bajo condiciones extraordinarias.
  % \item Los exámenes podrán reponerse con el visto bueno del director de carrera, quien deberá enviar una notificación al profesor (un correo, por ejemplo).
  \item Es tu responsabilidad ponerte al tanto de lo acontecido en la clase durante tu ausencia.
  \item No faltes a clase si no es absolutamente necesario, pues solemos ir bastante rápido en este curso.
  \item Sé cortés durante la sesión. Se recomienda que prendas tu cámara y silencies el micrófono al entrar. En las discusiones, tomaremos turnos para participar. Asegúrate de que tu celular está en silencio si tu micrófono está abierto. Si recibes una llamada o mensaje importante durante una sesión, podrás atenderlo sin problemas pero asegúrate de que el micrófono está desactivado.\footnotemark
  \end{itemize}

  \footnotetext{El problema principal no es que tú no te concentres, sino que podrías perjudicar al ambiente en que se desenvuelven tus demás compañeros. Sé considerado.}

  \vskip.15in
  \noindent\textbf{Integridad académica:}
  ``Se entiende por \textit{integridad académica} el actuar honesto, comprometido, confiable, responsable, justo, respetuoso con el aprendizaje, la investigación y la difusión de la cultura''. En este curso, pedimos que los alumnos y el profesor se comporten siguiendo estos principios.
  \\[2ex]
  {\color{darkred}{\Large \HandRight}} ~\textbf{La copia en exámenes,tareas o evidencia va en forma flagrante contra dicha \textit{integridad académica}, y será penalizada}.
  Una cosa es \textit{hacer la tarea juntos} y otra muy distinta es compartir resultados y documentos sin hacer referencia formal de ello.\\[2ex]
  {\color{darkred}{\Large \HandRight}} ~El nuevo reglamento académico establece que el profesor asignará una \textbf{calificación reprobatoria} a la actividad, examen, período parcial o final. \textbf{La calificación reprobatoria asignada por el profesor será inapelable}, y a esta sanción se sumarán aquellas otras que el Comité de Integridad Académica del Campus determine pertinentes.

  %%%%%% END 
\end{document} 